\documentclass[a4paper,12pt,twoside]{article} % kaksipuolinen
\usepackage[finnish]{babel}            %suomenkielinen tavutus ja sanasto
\usepackage[T1]{fontenc}               %valitaan ääkkösfonttikoodaus
\usepackage[utf8]{inputenc}        	% skandit utf-8 koodauksella
\usepackage{graphicx}           %kuvat
%\usepackage[dvips]{graphicx}           %ps-kuvat

\setlength{\oddsidemargin}{1.85cm} %kaksipuoliset marginaalit
\setlength{\evensidemargin}{0.35cm} %kaksipuoliset marginaalit

\begin{document}

\pagestyle{empty}  %ei sivunumeroa sivun alareunaan

\begin{center}
\includegraphics[width=4cm]{soihtu.png} %talleta kuva linkistä omaan hakemistoosi
\end{center}

\vspace{3.0cm}
\begin{center}\large
Transferability of leadership and management
\end{center}

\vspace{0.5cm}
\begin{center}
Juha Ylönen
\end{center}

\vspace{0.5cm}
\begin{center}
Masters Thesis\\
December 2018
\end{center}

\vspace{5.0cm}
\begin{center}
Management and Organization\\
University of Turku\\
\end{center}

\cleardoublepage

%säädetään 2-puolisen sivun marginaalit vaatimuksia vastaaviksi
\addtolength{\oddsidemargin}{-1.1cm}
\addtolength{\voffset}{-2.3cm}

\begin{minipage}{15cm}

\noindent
University of Turku\\
Management and Organization\\
\\
JUHA YLÖNEN: Tiivistelmäsivun laadinta LaTeX-ladontaohjelmalla\\
Masters Thesis, n pages, n attachments.\\
Management and Organization\\
December 2018\\
\rule{\textwidth}{.2mm}\\
\\
Tiivistelmäsivun tekeminen LaTeX-ladontajärjestelmällä on helppoa. Kopioidaan esimerkiksi www-sivulta tiivistelmäsivun LaTeX-koodi, jossa on valmiiksi oikeanlaiset marginaalit.

\vspace{4mm}\noindent Sitten vain kirjoitellaan erittäin fiksua tekstiä tiivistelmäsivu täyteen. Muistetaan laittaa \noindent-komento kappaleen
alkuun, niin LaTeX ei sisennä ensimmäistä riviä.

\vspace{4mm}\noindent Tuo $\backslash$vspace-komento taasen jättää sopivan välin kappaleiden väliin - 4mm näyttää aika hyvältä.

\vspace{4mm}\noindent Muistetaan lopettaa ajoissa ympäripyöreän liirumlaarumin kirjoittelu.

\vspace{4mm}\noindent Asiasanat: tiivistelmäsivu, Pro gradu -tutkielma, LaTeX-ladontajärjestelmä.

\end{minipage}

\cleardoublepage

\tableofcontents
\clearpage

\pagestyle{plain} 
\setcounter{page}{1}

%palautetaan 2-puolisen sivun marginaalit oletusasetuksiin
\addtolength{\oddsidemargin}{1.1cm}
\addtolength{\voffset}{2.3cm}

\section{Introduction}
\label{Introduction}

Transferability of employee skills is a question of growing interest in a world
where workforce mobility is becoming a standard practice. The days of entering a
company straight from school and leaving only for retirement are long gone. In
this environment it is increasingly important to evaluate the impact of
epmloyees moving around in their effectiviness. The transferability aspect has
been studied [TODO: lähteitä] with regrads of star performing employees and
teams, but somewhat lesser attention has been given to transferability of the
managers and superiors. As managing and leading can be seen as sort of expert
profession in similar vein as programmer or engineer with skillset that can be
attained, should it be any different in transferability. 

This study will look into the transferability of leadership and management
skills. The study is based on data collected by the University of Turku
questionnaire for new managers that have recently started in management
positions with direct subordinates. Study will try to answer the question
whether management and leadership skills are transferable, i.e. if there is a
difference in the management-score of managers who are in their first superior
position and those who have held similar positions before by using statistical
methods. The study will try to formulate and calculate the previously mentioned
'management-score' by crossreferencing and combining several well-known
management scorecards and tailor these to fit the data in question. Secondary
question will be comparing the difference in the scores of managers who have
held superior positions before, but either inside the same company or elsewhere.
This should give us a look into the locality (TODO: tsekkaa oikea termi ja
lähde) of management and leadership skills, i.e. the transferable portion of
said skills. Tertial interest is in the difference of the scores of superiors
categorized either as 'Leader' or 'Manager' (TODO: lähde) in order to
determinate the difference of transferability of either leadership or
management ability. 

The study will be statistical analysis of the collected data by using R
programming language and several methods of statistical analysis combined. As
the result of analysis the managers participating the study are categorized in
several categories (new-experienced, local-external, leader-manager) and rated
in their effectiveness as a manager.

Next section of the paper will form the theoretical background for the study,
with chapters for transferability, distinction between manager and leader and
available management (and leadership) scorecards and rating systems. The third
section contains chapters that will describe in detail the dataset, the selected
tools for analysis and the methods formulated for categorization and rating. 
Fourth section will present the findings of the study and how the questions are
answered in the correlations that may or may not have been surfaced. Fifth and
final section will summarize the study and discuss some possibilities for
further study.


\clearpage
\section{Literature}
\label{Literature}

\subsection{Transferability of employees}
\label{Transfrrability of employees}
-overall description of tranferability theory, how it should affect performance
etc.
\\
-impact of local knowledge vs. subject knowledge etc

\subsection{Of managers and leaders}
\label{Of managers and leaders}
- what is the distinction of manager and leader. why it matters in this study
\\
- management more of local knowledge and leading as subject knowledge??? any
support? 

\subsection{Manager scorecards}
\label{Manager scorecards}
-what scorecards and evaluation methods are available and how they are used.
360degree? etc


\clearpage
\section{Method}
\label{Method}

\subsection{Dataset}
\label{Dataset}
- the background of the questionnaire, the data amounts etc. 

\subsection{Tools}
\label{Tools}
-R language etc. explained shortly

\subsection{Selected methods}
\label{Selected methods}

\subsubsection{Categories}
\label{Categories}
-how the categories will be formed, 1) with experience-no experience 2) from
inside company-from outside company (from category 1 group with experience) 3)
leader-manager (from category 1 group with experience) 

\subsubsection{Scorecard}
\label{Scorecard}
-what variables are used and how. How the final value is calculated, what
variables are emphasised

\clearpage
\section{Results}
\label{Results}

\subsection{Validity of findings}
\label{Validity of findings}
-what were the statistical assumptions made and how they might have affeted the
results etc.

\subsection{Research questions answered}
\label{Research questions answered}
-how the correlations answered the questions

\subsubsection{Transferability of management and leadership skills}
\label{Transferability of management and leadership skills}
-if the scores of managers with previous experience were better

\subsubsection{Impact of local knowledge}
\label{Impact of local knowledge}
-if the scores of managers who've been promoted or transfered inside company
were better than those hired from outside (when they had previous experience)

\subsubsection{Impact of leadership or management tendency}
\label{Impact of leadership or management tendency}
-if there were differences in performances of leaders vs. managers in previous
categories (when they had previous experience)

\clearpage
\input{conclusion.tex}
\clearpage

% insert references
\bibliographystyle{unsrt}
\bibliography{bib.bib}



\end{document}
